\chapter{Wstęp}\label{Chapter_Wstep}

  \section{Motywacja}\label{Section_Motywacja}

  \section{Cel}\label{Section_Cel}

\chapter{Problem śledzenia punktów charakterystycznych w~sekwencjach wideo}

  \section{Opis dziedziny problemu}\label{Section_Problematyka}
    Mając do czynienia z~sekwencjami wideo, przekształconymi do serii obrazów, bardzo częstym wymaganiem jest możliwość śledzenia punktu przez cały czas trwania sekwencji. Zagadnienie to rozszerza koncepcję wyodrębnienia cech charakterystycznych ze~statycznego obrazu, ponieważ wymaga analizy ruchu i~trajektorii śledzonego obiektu. Takie rozszerzone podejście wymaga dwuetapowego przetwarzania podzielonego na \textit{identyfikację punktów charakterystycznych} i~\textit{modelowanie ruchu śledzonych punktów charakterystycznych}.

    Problem identyfikacji punktów charakterystycznych polega na znalezieniu fragmentu obrazu w~aktualnej ramce, o~pewnej ważnej dla algorytmu charakterystyce, i~zidentyfikowanie go w~kolejnej klatce sekwencji wideo. Problemem powiązanym z~opisywanym zagadnieniem jest śledzenie niezidentyfikowanych jeszcze elementów ramki sekwencji wideo, których ważna z~punktu widzenia śledzenia i~algorytmu charakterystyka związana jest z~ruchem, lub dokładniej, określonymi parametrami związanymi z~poruszaniem się obiektu (kierunek, szybkość etc.). Wspomniane podejście wymaga śledzenia punktów wizualnie się odznaczających (inaczej \textit{wybitnych} lub \textit{charakterystycznych}) a~nie całych obiektów.

    Drugi etap polegający na modelowaniu ruchu śledzonych punktów jest pomocny przy próbie wyodrębnienia szczegółowych informacji na temat trajektorii, obwiedni lub kształtu śledzonych punktów z~często zniekształconych, zaszumionych danych wyznaczonych przez etap pierwszy. W~większości przypadków do wyznaczenia trajektorii i~np. przewidywania dalszego ruchu niezbędny jest złożony aparat matematyczny zaaplikowany w~przestrzeni dwu lub trójwymiarowej (w~zależności od charakteru danych wejściowych i~wymagań dotyczących analizowanego elementu).

  \section{Podstawowe definicje}\label{Section_Definicje}
    Zanim opisany zostanie problem doboru cech dla algorytmów śledzenia punktów charakterystycznych, należy wprowadzić zbiór definicji, w~celu ujednolicenia słownictwa wykorzystywanego w~pracy.

    \textbf{Klatką} (lub \textit{ramką}) nazywamy macierz kolorów (w~takim przypadku pojedynczym elementem składowym macierzy jest wektor trójelementowy lub czteroelementowy, zwany dalej \textit{krotką kolorów}) lub intensywności (wtedy pojedynczym elementem składowym jest liczba rzeczywista z~obustronnie domkniętego przedziału $<0.0, 1.0>$). Parametrami charakterystycznymi ramki są jej wymiary (\textit{wysokość} i~\textit{szerokość}) określone z~dokładnością do pojedynczego piksela oraz \textit{głębia kolorów}.

    \textbf{Sekwencją wideo} (lub \textit{animacją}) nazywamy skończony zbiór klatek o~identycznych wymiarach i~identycznej głębi kolorów. Oprócz wspomnianych elementów animacja posiada również \textbf{czas trwania} określony w~sekundach (lub jednostkach pochodnych) oraz \textbf{długość sekwencji wideo} określona za pomocą mocy zbioru ramek.

    Kolejną z~wartości charakterystycznych sekwencji wideo jest ilość \textbf{ramek na sekundę} (ang. \textit{Frames Per Second}, FPS), która nie musi być tożsama z~\textit{częstotliwością przetwarzania sekwencji wideo}.

    Aby można było w~pełni mówić o~częstotliwości musimy wprowadzić definicję \textbf{obrazu}. Wyróżniamy dwa tryby wyświetlania obrazów: tryb \textit{progresywny} (oznaczenie \textit{p}, ang. \textit{progressive}) i~tryb z~\textit{przeplotem} (oznaczenie \textit{i}, ang. \textit{interlaced}). W~przypadku trybu progresywnego (stosowanego w~przypadku współczesnego sprzętu tj. monitorów innych niż katodowe \textit{CRT} i~plazmowe \textit{ALiS}) ilość obrazów jest równa ilości klatek. Dlatego w~przypadku tego trybu wartość \textit{ramek na sekundę}, może zostać zapisana z~nominalną jednostką częstotliwości z~układu SI. Odwrotność częstotliwości przetwarzania określa \textbf{czas przetwarzania pojedynczej klatki} analizowanej sekwencji.

    Dla przykładu, sekwencja wideo o~\textit{ilości ramek na sekundę} równej 24 \textit{FPS} posiada częstotliwość: \[ f_{wideo} = 24 [FPS] = 24 [Hz] \] a~z~tej wartości wynika następujący czas przetwarzania pojedynczej ramki: \[ t_{przetwarzania} = \frac{1}{24 [Hz]} = 41,67 [ms] \]

    \textbf{Punktem charakterystycznym} nazywamy punkt ramki animacji wraz z~jego otoczeniem (tzw. \textit{okno}, \cite{SalientPointsTracking05}) niezbędnym do wyznaczenia parametrów wyróżniających i~umożliwiających śledzenie. Rozmiar okna oraz sposób wyznaczania własności charakteryzujących dany punkt i~jego wybitność został dokładnie omówiony w~rozdziale \ref{Section_GoodGeaturesToTrack}.

    \textbf{Punktem kluczowym} nazywamy pośredni punkt wyznaczający ważną część pokonywanej trasy, lub kształtu. Pomiar odległości w~metryce euklidesowej między punktem kluczowym a~najbliższym śledzonym punktem charakterystycznym jest podstawowym parametrem jakościowym wykorzystywanym do porównania algorytmów między sobą (dokładny opis znajduje się w rozdziale~\ref{Section_Jakosc}).

    \textbf{Połączeniem} nazywamy odcinek w~metryce euklidesowej łączący dwa wybrane punkty kluczowe. Najmniejsza odległość wśród śledzonych punktów charakterystycznych od aktualnego połączenia jest kolejną z~wartości charakteryzujących jakość algorytmu.

    \textbf{Kształtem} (lub \textit{trajektorią}, \textit{obwiednią}) nazywamy zbiór punktów kluczowych połączonych za pomocą wspomnianych wyżej odcinków. Sama definicja poza nazwą i~pewną wzrokową wizualizacją nie niesie bezpośrednio ze sobą żadnych informacji jakościowych. Pośrednio, ostre lub gładkie zakończenia oraz rogi stworzonych kształtów pozwalają na wizualne zbadanie jakości trajektorii wyznaczonej przez śledzone punkty i~odstępstw od oryginalnego kształtu.

  \section{Problem doboru cech dla algorytmów śledzenia punktów charakterystycznych}\label{Section_GoodGeaturesToTrack}
    Wybór punktów charakterystycznych dla problemu śledzenia ich w~sekwencji wideo jest kluczowym elementem z~punktu widzenia jakości wyznaczonego rozwiązania końcowego. Jeśli wyobrazimy sobie sekwencję wideo jedynie ze strukturą o~powtarzalnej teksturze, to wybór charakterystycznej elementu okaże się bardzo trudny. Wiele punktów będzie identycznych lub bardzo podobnych do siebie i~śledzenie jednego wybranego pomiędzy kolejnymi ramkami animacji jest z~góry skazane na porażkę. Z~drugiej strony, jeśli mamy do dyspozycji unikalny fragment, wybierając go mamy duże szanse odnaleźć go w~następnych ramkach.

    Podejście intuicyjne do tego problemu, polegające na poszukiwaniu odróżniającego się szczegółu jest krótkowzroczne. Przy wyborze charakterystycznej cechy możemy się kierować dużą wartością pochodnej w~punkcie - tego typu wartości oznaczają najczęściej \textbf{krawędź}. Niestety, podstawowa cechą \textit{krawędzi} jest to, że wiele punktów leżących na niej posiada identyczne wartości (prowadzi to do tzw. "problemu szczeliny", który omówiony został poniżej). Potrzebne są bardziej unikalne właściwości.

    Pierwsze podejście do problemu doboru cech, pozwoliło nam odkryć bardzo ważną własność dotyczącą wartości pochodnej. Okazuje się, że jeśli wartości pochodnej w~dwóch równoległych do siebie kierunkach są duże, to dany punkt jest dużo bardziej unikalny w~skali całej ramki - nazywamy go wtedy \textbf{narożnikiem}.

    Najpowszechniej używana definicja \textit{narożnika} została przygotowana przez \textit{Harrisa} (szczegóły zawarte w~pracy \cite{Harris88}) i~z~niej również korzystają autorzy w~pracy \cite{GoodFeaturesToTrack94}.

    Definicja opiera się macierzy drugich pochodnych intensywności obrazu w~określonym punkcie ($\partial^2 x$, $\partial^2 y$, $\partial x\partial y$). W~szczególności możemy obliczyć wartości wartości wspominanych pochodnych dla całej ramki animacji tworząc w~ten sposób nową ramkę przechowującą wartości drugich pochodnych nazwaną \textit{obrazem Hesjanów}. Hesjan (lub dokładniej \textit{macierz Hessego}) definiujemy następująco ($I$ jest intensywnością piksela $p$ obrazu):

    \[
      H(p) =
        \begin{bmatrix}
          \frac{\partial^2 I}{\partial x^2} & \frac{\partial^2 I}{\partial x\partial y} \\
          \frac{\partial^2 I}{\partial x\partial y} & \frac{\partial^2 I}{\partial y^2} \\
        \end{bmatrix}_{p}
    \]

    Dla powyższej definicji narożnika kolejnym ważnym elementem jest \textit{macierz autokorelacji drugich pochodnych} dla każdego śledzonego punktu otoczonego \textit{oknem} o~określonych, małych wymiarach. Macierz autokorelacji jest zdefiniowana w~następujący sposób ($w_{i,j}$ jest współczynnikiem zależnym od zastosowanego okna):

    \[
      M(x, y) =
        \begin{bmatrix}
            \displaystyle\sum\limits_{K \le i, j \le K}
              \scalemath{0.75}{w_{i,j} I_{x}^{2} (x + i, y + j)} &
            \displaystyle\sum\limits_{K \le i, j \le K}
              \scalemath{0.75}{w_{i,j} I_{x} (x + i, y + j) I_{y} (x + i, y + j)} \\

            \displaystyle\sum\limits_{-K \le i, j \le K}
              \scalemath{0.75}{w_{i,j} I_{x} (x + i, y + j) I_{y} (x + i, y + j)} &
            \displaystyle\sum\limits_{-K \le i, j \le K}
              \scalemath{0.75}{w_{i,j} I_{y}^{2} (x + i, y + j)} \\
        \end{bmatrix}
    \]

    Przy tak zdefiniowanej \textit{macierzy autokorelacji} narożnikiem zgodnie z~definicją \textit{Harrisa} nazywamy miejsce, dla którego powyższa macierz posiada \textit{dwie duże wartości własne}. Oznacza to, że w~danym punkcie istnieje krawędź (lub faktura) rozciągająca się w~dwóch kierunkach, tak jak w~przypadku rzeczywistego narożnika. Warto zauważyć, że dzięki zastosowaniu drugiej pochodnej całość jest niewrażliwa na stałą wartość pierwszej pochodnej (druga pochodna będzie wtedy równa $0$).

    Zastosowanie wartości własnych dla tej definicji ma kilka dodatkowych zalet. Dzięki wybraniu miejsc posiadających duże wartości własne (a~w konsekwencji wektory własne) zyskujemy niewrażliwość wybranego fragmentu na rotację. Dodatkowo dwie duże wartości własne, stanowią same w~sobie bardzo dobre wartości do śledzenia w~przyszłych klatkach animacji.

    Badając definicję \textit{Harrisa} autorzy pracy \cite{GoodFeaturesToTrack94}, \textit{Shi} oraz \textit{Tomasi}, zauważyli, że dobrym \textit{narożnikiem} (w~sensie jakościowym) można nazwać taki element, którego mniejsza z~obliczonych wartości własnych jest zawsze większa od pewnego przyjętego \textit{minimalnego progu}. Taki sposób wyznaczania narożników daje w~większości przypadków dużo lepsze rezultaty niż oryginalna definicja. W~rozdziale \ref{Section_ImplementationDetails} została omówiona dokładnie procedura $goodFeaturesToTrack()$ korzystającą z~powyżej własności.

    Inną bardzo ważną definicją jest \textit{Skaloniezmiennicze przekształcenie cech} (ang. \textit{Scale-invariant feature transform}, w skrócie \textit{SIFT}) opisane w~pracy \cite{SalientPointsTracking05}. Pozwala ona na wykrycie cech w~obrazie, które są odporne na skalowanie, rotację i~przesunięcie. Omawiane cechy obliczane są na podstawie kierunku dominującego gradientu w~określonym położeniu i~naniesieniu wartości do lokalnego histogramu na podstawie jego orientacji. Dzięki temu, przy odpowiednio małych przekształceniach afinicznych (w~omawiamym przypadku, pomiędzy dwoma bezpośrednio sąsiadującymi klatkami sekwencji wideo) obserwowane cechy nie zmieniają się.

\chapter{Analiza i~porównanie wybranych algorytmów}\label{Section_Algorytmy}

  \section{Algorytmy oparte o~przepływ optyczny}\label{Subsection_OpticalFlow}

    \subsection{Ogólny zarys algorytmów opartych o~przepływ optyczny}
    Przy śledzeniu punktów charakterystycznych, bardzo często nie posiadamy szczegółowej wiedzy o~środowisku i~obserwowanym podmiocie. W~takiej sytuacji własności algorytmów opartych o~przepływ optyczny sprawdzają się najlepiej, ponieważ sam ruch i~zmiany zachodzące pomiędzy kolejnymi klatkami animacji same w~sobie niosą najwięcej informacji\cite{OpticalFlowNonPriori05}.

    Przypisując wektor prędkości (lub przemieszczenie, różnicę odległości jaką piksel przebył od swojego ostatniego położenia w~poprzedniej ramce sekwencji wideo) do każdego piksela, tworzymy obraz przepływu optycznego nazwany \textit{gęstym} (został on szczegółowo omówiony w~\ref{Subsection_DenseOpticalFlow}). Powstały obraz nazywamy \textit{wektorowym polem prędkości}. Podczas obserwacji tego typu metod, łatwo zauważyć, że wiele wektorów prędkości na obrazie posiada długość $0$ (innymi słowy, piksele w~tych miejscach się nie poruszają) i~można pokusić się o~optymalizację metody gęstej poprzez wprowadzenie okna obserwacji nazwanego \textit{blokiem}. Taki rodzaj gęstego przepływu optycznego nazywamy \textit{dopasowywaniem bloków}, nie został on jednak wzięty pod uwagę w~przeprowadzonych badaniach.

    Obliczenie \textit{wektorowego pola prędkości} jest zadaniem złożonym obliczeniowo. Złożoność asymptotyczna wynosi $O(Cnm)$, gdzie $C$ jest stałą związaną z~obliczeniami wektora prędkości dla danego punktu i~$n$ oraz $m$ są wymiarami klatki animacji. Omawiana stała posiada bardzo duży wpływ na końcową postać złożoności ze~względu na zastosowanie metody interpolacji między śledzonymi punktami w~celu uzyskania jednoznacznego rezultatu.

    Gęsty przepływ optyczny, mimo intuicyjnej definicji, jest metodą o~bardzo dużym koszcie obliczeniowym. Próba zredukowania obliczanego pola wektorów doprowadziła do stworzenia metody \textit{dopasowania bloków} omówionej powyżej. Intuicyjnie, możliwe jest jeszcze większe ograniczenie złożoności, poprzez śledzenie tylko charakterystycznej grupy punktów, poprzez zastosowanie \textit{definicji narożnika} omówionej szczegółowo w~rozdziale \ref{Section_GoodGeaturesToTrack} oraz artykułach \cite{LucasKanadeTracker81} i~\cite{GoodFeaturesToTrack94}. Właśnie takie sformułowanie problemu nazywane jest \textit{rzadkim przepływem optycznym}. Koszt obliczeniowy jest dużo mniejszy niż w~przypadku gęstego przepływu i~jest wprost proporcjonalny do mocy zbioru punktów charakterystycznych, która w~zdecydowanej większości przypadków jest dużo mniejsza od ilości punktów obrazu.

    Zupełnie innym podejściem jest zastosowanie...

    W~kolejnych sekcjach zostały szczegółowo omówione algorytmy przepływu optycznego rzadkiego, gęstego oraz metoda oparta o~całowitą zmienność i~normę L1.

    \subsection{Rzadki przepływ optyczny (algorytm Lucas-Kanade)}
    \cite{OpenCvOpticalFlow04}

    \subsection{Gęsty przepływ optyczny (algorytm Farnebäcka)}\label{Subsection_DenseOpticalFlow}
    \cite{GunnarFarneback03}

    \subsection{Przepływ optyczny $TV-L_{1}$}
    \cite{TVL107}

  \section{Algorytmy oparte punkty kluczowe}\label{Subsection_SaliencyKeypoints}
    \cite{SalientPointsTracking05}

  \section{Algorytm oparty o~filtr cząsteczkowy}\label{Subsection_ParticleFilter}
    \cite{ParticleFilter05}
    \cite{FingertipParticleFilter11}

  \section{Algorytm oparty o~decyzyjne drzewa losowe}\label{Subsection_RandomizedTrees}
    \cite{RandomizedTrees06}
    \cite{TwoStageRandomizedTrees11}

\chapter{Specyfikacja przeprowadzonych badań}\label{Chapter_SpecyfikacjaPrzeprowadzonychBadan}

  \section{Definicje gestów}\label{Section_DefinicjeGestow}

    W~ramach przeprowadzonych badań stworzono definicje poszczególnych gestów wykonywanych dłońmi użytkownika. Poniższa specyfikacja ma na celu uwspólnienie pojęć potrzebnych przy przeprowadzeniu analizy jakościowej oraz wydajnościowej.

    Wytyczne były pomocne przy procesie akwizycji oraz obróbki danych wejściowych. Wszystkie stworzone materiały testowe są zgodne z~przedstawionymi protokołami. W~zbiorze danych wejściowych możemy wyróżnić następujące definicje gestów:

    \gest{Okrąg}
         {Gest wykonywany na tle twarzy lub ciała.}
         {Okrąg o~promieniu 20 cm.}
         {Prawa dłoń.}
         {Okrąg wykonywany do kamery otwartą dłonią.}
         {Kierunek ruchu jest zgodny z~ruchem wskazówek zegara.}
         {Dłoń znajduje się na godzinie dwunastej.}
         {Punkty charakterystyczne rozmieszczone równomiernie na krawędziach dłoni.}
         {Punkty kluczowe połączone linią ciągłą rozmieszczone na okręgu co 30\degree.}

    \newpage
    \gest{Krzyż}
         {Gest wykonywany na tle twarzy lub ciała.}
         {Krzyż równoramienny o~promieniu 7 cm.}
         {Dwa, wyciągnięte palce (wskazujący oraz środkowy) prawej dłoni}
         {Kształt zarysowywany zewnętrzną strona dłoni.}
         {W~pierwszej kolejności wykonywane jest ramię pionowe, następnie palce powracają do środka i~gest wykonywany jest w~lewą stronę a~następnie w~prawo.}
         {Ustawiona dłoń znajduje się na godzinie dwunastej.}
         {Punkty charakterystyczne rozmieszczone na czubkach wyciągniętych palców.}
         {5 punktów kluczowych - cztery na krawędziach i~jeden w~środku figury.}

    \gest{Rozszerzanie}
         {Gest wykonywany w~pozycji siedzącej.}
         {W~pozycji wyjściowej, jest to kwadrat o~długości boku około 20 cm.}
         {Obie dłonie złączone czubkami palców uformowane w~gest \textit{szczypty}.}
         {Oburęczny gest rozciągania dłoni od środka.}
         {Gest powinien symulować rozciąganie niewidzialnego materiału. Kształt dłoni nie powinien ulegać zmianie w~trakcie wykonywania gestu.}
         {Obie złączone dłonie znajdują się w~środku figury na wysokości klatki piersiowej.}
         {Punkty charakterystyczne rozmieszczone równomiernie na powierzchni obu dłoni.}
         {Punkty kluczowe równomiernie rozmieszczone na przekątnej kwadratu.}

    \newpage
    \gest{Duża litera C}
         {Gest wykonywany na jednolitej i~gładkiej powierzchni.}
         {Półokrąg o~promieniu 10 cm.}
         {Palec wskazujący lewej dłoni.}
         {Zarysowanie kształtu w~kierunku przeciwnym do ruchu wskazówek zegara.}
         {Gest powinien być wykonywany przez poruszenie całej dłoni, nie tylko samego palca.}
         {Palec wskazujący na godzinie dwunastej.}
         {Punkty charakterystyczne zaznaczone na krawędziach analizowanego palca.}
         {Punkty połączone linią ciągłą rozmieszczone równomiernie na półokręgu co 30\degree.}

    \gest{Duża litera L}
         {Gest wykonywany na jednolitej i~gładkiej powierzchni.}
         {Litera "L" o~wysokości 7 cm.}
         {Złączone palce prawej dłoni.}
         {Kształt litery zarysowują czubki palców.}
         {Gest powinien być wykonywany przez poruszanie tylko złączonymi palcami, nie samą dłonią.}
         {Palce na godzinie dwunastej.}
         {Punkty charakterystyczne rozmieszczone równomiernie na górnej części dłoni.}
         {4 punkty kluczowe - na początku, na końcu, na zgięciu litery oraz dodatkowy punkt na dłuższej części litery, dokładnie w~jej połowie.}

    \newpage
    \gest{Zgniecenie}
         {Gest wykonywany na tle twarzy lub ciała.}
         {Zaciśnięcie dłoni od rozwarcia do pięści.}
         {Prawa dłoń.}
         {W~pozycji wejściowej otwarta dłoń, w~pozycji końcowej pięść zwrócona palcami do kamery, z~kciukiem na wierzchu.}
         {Gest powinien być wykonywany bez poruszania dłonią, z~jednolitą prędkością.}
         {Otwarta dłoń z~rozłożonymi palcami, skierowana wewnętrzną stroną do kamery.}
         {Punkty charakterystyczne znajdują się na czubkach palców.}
         {Punkty kluczowe rozmieszczone równomiernie na drodze wykonywanej przez każdy z~palców z~osobna.}

  \section{Akwizycja i~obróbka danych wejściowych}\label{Section_Akwizycja}

  \section{Procedura weryfikacji jakości badanego algorytmu}\label{Section_Jakosc}

    W~celu analizy i~porównania jakości poszczególnych algorytmów przygotowano zestaw parametrów, które będą badane podczas wykonania każdego z~nich. Główną rolę odgrywają tutaj \textbf{punkty kluczowe} oraz \textbf{połączenia} między nimi.

    Jakość algorytmu wyznaczana będzie na podstawie odchyleń tras oraz pozycji punktów charakerystycznych w~stosunku do pozycji punktów kluczowych oraz tras pomiędzy nimi.

    Każdy z~gestów posiada serię punktów kluczowych połączonych ze sobą liniami prostymi (w~celu symulacji gładszych połączeń np. okręgu stosuje się odpowiednią ilość punktów pośrednich). Każdy z~punktów kluczowych oprócz swojej pozycji posiada również \textit{numer klatki wideo} w~której występuje.

    W~określonych przez punkty kluczowe klatkach animacji następuje przeszukanie wszystkich dostępnych w~danej chwili czasu \textit{t} punktów charakterystycznych i~obliczana jest ich odległość od punktu kluczowego. Do porównania wybierana jest nabliższy punkt zgodnie z~metryką euklidesową.

    W~klatkach pośrednich analizowana jest odległość wszystkich punktów charakterystycznych od aktualnej \textit{trasy}. Dzięki temu, wiemy jak wygląda dystrybucja punktów charakterystycznych w~okół śledzonej trasy.

    Warto zauważyć, że taki sposób weryfikacji odporny jest na zmiany (poprzez dokładanie lub usuwanie) punktów charakerystycznych.

    Biorąc pod uwagę dwa wspomniane parametry tj. \textbf{odchylenie minimalne od punktu kluczowego w~danej klatce sekwencji wideo} oraz \textbf{wektor odchyleń punktów charakterystycznych od trasy} możemy skutecznie porównać jakość badanych algorytmów między sobą.

  \section{Procedura weryfikacji wydajności badanego algorytmu}\label{Section_Wydajnosc}

    W~celu analizy i~porównania wydajności poszczególnych algorytmów przygotowano zestaw parametrów, które obliczane będą podczas wykonania każdego algorytmu. Ujednolicenie to ma celu wypracowanie wspólnego wzorca porównawczego różnych algorytmów o~zupełnie innej zasadzie działania oraz implementacji.

    Główne parametry porównawcze opierają się na pomiarach czasu wykonania oraz własnościami statystycznymi wyciągniętymi z~pomiarów.

    Pierwszy pomar to \textbf{czas pracy algorytmu na pojedynczej klatce}. Z~wartości pomiarów dla wszystkich klatek sekwencji wideo wyciągnięte zostaną własności statystyczne tj. \textit{średnia arytmetyczna}, \textit{mediana}, \textit{odchylenie standardowe}.

    Kolejnym pomiarem jest \textbf{sumaryczny czas przetwarzania algorytmu}. Warto zauważyć, że nie będzie on nigdy mniejszy od nominalnego czasu trwania sekwencji wideo. Sumaryczny czas może być jedynie równy wartości nominalnej w~przypadku, gdy przetwarzanie pojedynczej klatki wideo odbędzie się w~czasie mniejszym bądź równym wartości niezbędnej do odczekania przed wyświetleniem następnej klatki. Czasu oczekiwania jest ściśle związany z~wartością \textit{częstotliwości klatek na sekundę} analizowanej sekwencji wideo.

    Następna wartość to \textbf{odchylenie sumarycznego czasu wykonania od czasu nominalnego} wyrażona w~procentach.

    Ostatni parametr to próba klasyfikacji wszystkich algorytmów według stopnia ich złożoności. Każdy algorytm będzie poddany analizie pod kątem wyselekcjonowanego zbioru operacji dominujących, w~celu wyznaczenia \textbf{złożoności obliczeniowej oraz pamięciowej}.