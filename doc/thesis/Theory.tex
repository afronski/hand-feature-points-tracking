\chapter{Wstęp}\label{Chapter_Wstep}

  \section{Motywacja}\label{Section_Motywacja}

  \section{Cel}\label{Section_Cel}

\chapter{Problem śledzenia punktów charakterystycznych w~sekwencjach wideo}

  \section{Opis dziedziny problemu}\label{Section_Problematyka}
    Gdy mamy do czynienia z~sekwencjami wideo, przekształconymi do serii obrazów, bardzo częstym wymaganiem jest możliwość śledzenia punktu przez cały czas trwania sekwencji. Zagadnienie to rozszerza koncepcję wyodrębnienia cech charakterystycznych ze~statycznego obrazu, ponieważ wymaga analizy ruchu i~trajektorii śledzonego obiektu. Takie rozszerzone podejście wymaga dwuetapowego przetwarzania podzielonego na \textit{identyfikację punktów charakterystycznych} i~\textit{modelowanie ruchu śledzonych punktów charakterystycznych}.

    Problem identyfikacji punktów charakterystycznych polega na znalezieniu fragmentu obrazu w~aktualnej ramce, o~pewnej ważnej dla algorytmu charakterystyce, i~zidentyfikowanie go w~kolejnej klatce sekwencji wideo. Problemem powiązanym z~opisywanym zagadnieniem jest śledzenie niezidentyfikowanych jeszcze elementów ramki sekwencji wideo, których ważna z~punktu widzenia śledzenia i~algorytmu charakterystyka związana jest z~ruchem, lub dokładniej, określonymi parametrami związanymi z~poruszaniem się obiektu (kierunek, szybkość etc.). Wspomniane podejście wymaga śledzenia punktów wizualnie się odznaczających (inaczej \textit{wybitnych} lub \textit{charakterystycznych}) a~nie całych obiektów.

    Drugi etap polegający na modelowaniu ruchu śledzonych punktów jest pomocny przy próbie wyodrębnienia szczegółowych informacji na temat trajektorii, obwiedni lub kształtu śledzonych punktów z~często zniekształconych, zaszumionych danych wyznaczonych przez etap pierwszy. W~większości przypadków do wyznaczenia trajektorii i~np. przewidywania dalszego ruchu niezbędny jest złożony aparat matematyczny zaaplikowany w~przestrzeni dwu lub trójwymiarowej (w~zależności od charakteru danych wejściowych i~wymagań dotyczących analizowanego elementu).

  \section{Podstawowe definicje}\label{Section_Definicje}
    Zanim opisany zostanie problem doboru cech dla algorytmów śledzenia punktów charakterystycznych, należy wprowadzić zbiór definicji, w~celu ujednolicenia słownictwa wykorzystywanego w~pracy.

    \textbf{Klatką} (lub \textit{ramką}, \textit{obrazem}) nazywamy macierz kolorów (w~takim przypadku pojedynczym elementem składowym macierzy jest wektor trójelementowy lub czteroelementowy, zwany dalej \textit{krotką kolorów}) lub intensywności (wtedy pojedynczym elementem składowym jest liczba rzeczywista z~obustronnie domkniętego przedziału $<0.0, 1.0>$). Parametrami charakterystycznymi ramki są jej wymiary (\textit{wysokość} i~\textit{szerokość}) określone z~dokładnością do pojedynczego piksela oraz \textit{głębia kolorów}.

    \textbf{Sekwencją wideo} (lub \textit{animacją}) nazywamy skończony zbiór klatek o~identycznych wymiarach i~identycznej głębi kolorów. Oprócz wspomnianych elementów animacja posiada również \textbf{czas trwania} określony w~sekundach (lub jednostkach pochodnych) oraz \textbf{długość sekwencji wideo} określona za pomocą mocy zbioru ramek.

    Kolejną z~wartości charakterystycznych sekwencji wideo jest ilość \textbf{ramek na sekundę} zwana również \textit{częstotliwością przetwarzania sekwencji wideo} (ang. \textit{Frames Per Second}, FPS). Odwrotność częstotliwości przetwarzania określa \textbf{czas przetwarzania pojedynczej klatki} analizowanej sekwencji.

    Dla przykładu sekwencja wideo o~podanych parametrach: \[ f_{wideo} = 24 [FPS] = 24 [Hz] \] posiada następujący czas przetwarzania pojedynczej ramki: \[ t_{przetwarzania} = \frac{1}{24 [Hz]} = 41,67 [ms] \]



    \cite{SalientPointsTracking05}
    \cite{GoodFeaturesToTrack94}
    \cite{LucasKanadeTracker81}

  \section{Problem doboru cech dla algorytmów śledzenia punktów charakterystycznych}
    \cite{GoodFeaturesToTrack94}

\chapter{Analiza i~porównanie wybranych algorytmów}\label{Section_Algorytmy}

  \subsection{Algorytmy oparte o~przepływ optyczny}\label{Subsection_OpticalFlow}

    \subsubsection{Ogólny zarys algorytmów opartych o~przepływ optyczny}
    \cite{OpticalFlowNonPriori05}
    \cite{LucasKanadeTracker81}

    \subsubsection{Rzadki przepływ optyczny (algorytm Lucas - Kanade)}
    \cite{OpenCvOpticalFlow04}

    \subsubsection{Gęsty przepływ optyczny (algorytm Gunnara Farnebäcka)}
    \cite{GunnarFarneback03}

    \subsubsection{TV-L}
    \cite{TVL107}

  \subsection{Algorytmy oparte punkty kluczowe i~analizę wybitności}\label{Subsection_SaliencyKeypoints}
    \cite{SalientPointsTracking05}

  \subsection{Algorytm oparty o~filtr cząsteczkowy}\label{Subsection_ParticleFilter}
    \cite{ParticleFilter05}
    \cite{FingertipParticleFilter11}

  \subsection{Algorytm oparty o~decyzyjne drzewa losowe}\label{Subsection_RandomizedTrees}
    \cite{RandomizedTrees06}
    \cite{TwoStageRandomizedTrees11}

\chapter{Specyfikacja przeprowadzonych badań}\label{Chapter_SpecyfikacjaPrzeprowadzonychBadan}

  \section{Definicje gestów}\label{Section_DefinicjeGestow}

    W~ramach przeprowadzonych badań stworzono definicje poszczególnych gestów wykonywanych dłońmi użytkownika. Poniższa specyfikacja ma na celu uwspólnienie pojęć potrzebnych przy przeprowadzeniu analizy jakościowej oraz wydajnościowej.

    Wytyczne były pomocne przy procesie akwizycji oraz obróbki danych wejściowych. Wszystkie stworzone materiały testowe są zgodne z~przedstawionymi protokołami. W~zbiorze danych wejściowych możemy wyróżnić następujące definicje gestów:

    \gest{Okrąg}
         {Gest wykonywany na tle twarzy lub ciała.}
         {Okrąg o~promieniu 20 cm.}
         {Prawa dłoń.}
         {Okrąg wykonywany do kamery otwartą dłonią.}
         {Kierunek ruchu jest zgodny z~ruchem wskazówek zegara.}
         {Dłoń znajduje się na godzinie dwunastej.}
         {Punkty charakterystyczne rozmieszczone równomiernie na krawędziach dłoni.}
         {Punkty kluczowe połączone linią ciągłą rozmieszczone na okręgu co 30\degree.}

    \newpage
    \gest{Krzyż}
         {Gest wykonywany na tle twarzy lub ciała.}
         {Krzyż równoramienny o~promieniu 7 cm.}
         {Dwa, wyciągnięte palce (wskazujący oraz środkowy) prawej dłoni}
         {Kształt zarysowywany zewnętrzną strona dłoni.}
         {W~pierwszej kolejności wykonywane jest ramię pionowe, następnie palce powracają do środka i~gest wykonywany jest w~lewą stronę a~następnie w~prawo.}
         {Ustawiona dłoń znajduje się na godzinie dwunastej.}
         {Punkty charakterystyczne rozmieszczone na czubkach wyciągniętych palców.}
         {5 punktów kluczowych - cztery na krawędziach i~jeden w~środku figury.}

    \gest{Rozszerzanie}
         {Gest wykonywany w~pozycji siedzącej.}
         {W~pozycji wyjściowej, jest to kwadrat o~długości boku około 20 cm.}
         {Obie dłonie złączone czubkami palców uformowane w~gest \textit{szczypty}.}
         {Oburęczny gest rozciągania dłoni od środka.}
         {Gest powinien symulować rozciąganie niewidzialnego materiału. Kształt dłoni nie powinien ulegać zmianie w~trakcie wykonywania gestu.}
         {Obie złączone dłonie znajdują się w~środku figury na wysokości klatki piersiowej.}
         {Punkty charakterystyczne rozmieszczone równomiernie na powierzchni obu dłoni.}
         {Punkty kluczowe równomiernie rozmieszczone na przekątnej kwadratu.}

    \newpage
    \gest{Duża litera C}
         {Gest wykonywany na jednolitej i~gładkiej powierzchni.}
         {Półokrąg o~promieniu 10 cm.}
         {Palec wskazujący lewej dłoni.}
         {Zarysowanie kształtu w~kierunku przeciwnym do ruchu wskazówek zegara.}
         {Gest powinien być wykonywany przez poruszenie całej dłoni, nie tylko samego palca.}
         {Palec wskazujący na godzinie dwunastej.}
         {Punkty charakterystyczne zaznaczone na krawędziach analizowanego palca.}
         {Punkty połączone linią ciągłą rozmieszczone równomiernie na półokręgu co 30\degree.}

    \gest{Duża litera L}
         {Gest wykonywany na jednolitej i~gładkiej powierzchni.}
         {Litera "L" o~wysokości 7 cm.}
         {Złączone palce prawej dłoni.}
         {Kształt litery zarysowują czubki palców.}
         {Gest powinien być wykonywany przez poruszanie tylko złączonymi palcami, nie samą dłonią.}
         {Palce na godzinie dwunastej.}
         {Punkty charakterystyczne rozmieszczone równomiernie na górnej części dłoni.}
         {4 punkty kluczowe - na początku, na końcu, na zgięciu litery oraz dodatkowy punkt na dłuższej części litery, dokładnie w~jej połowie.}

    \newpage
    \gest{Zgniecenie}
         {Gest wykonywany na tle twarzy lub ciała.}
         {Zaciśnięcie dłoni od rozwarcia do pięści.}
         {Prawa dłoń.}
         {W~pozycji wejściowej otwarta dłoń, w~pozycji końcowej pięść zwrócona palcami do kamery, z~kciukiem na wierzchu.}
         {Gest powinien być wykonywany bez poruszania dłonią, z~jednolitą prędkością.}
         {Otwarta dłoń z~rozłożonymi palcami, skierowana wewnętrzną stroną do kamery.}
         {Punkty charakterystyczne znajdują się na czubkach palców.}
         {Punkty kluczowe rozmieszczone równomiernie na drodze wykonywanej przez każdy z~palców z~osobna.}

  \section{Akwizycja i~obróbka danych wejściowych}\label{Section_Akwizycja}

  \section{Procedura weryfikacji jakości badanego algorytmu}\label{Section_Jakosc}

    W~celu analizy i~porównania jakości poszczególnych algorytmów przygotowano zestaw parametrów, które będą badane podczas wykonania każdego z~nich. Główną rolę odgrywają tutaj \textbf{punkty kluczowe} oraz \textbf{połączenia} między nimi.

    Jakość algorytmu wyznaczana będzie na podstawie odchyleń tras oraz pozycji punktów charakerystycznych w~stosunku do pozycji punktów kluczowych oraz tras pomiędzy nimi.

    Każdy z~gestów posiada serię punktów kluczowych połączonych ze sobą liniami prostymi (w~celu symulacji gładszych połączeń np. okręgu stosuje się odpowiednią ilość punktów pośrednich). Każdy z~punktów kluczowych oprócz swojej pozycji posiada również \textit{numer klatki wideo} w~której występuje.

    W~określonych przez punkty kluczowe klatkach animacji następuje przeszukanie wszystkich dostępnych w~danej chwili czasu \textit{t} punktów charakterystycznych i~obliczana jest ich odległość od punktu kluczowego. Do porównania wybierana jest nabliższy punkt zgodnie z~metryką euklidesową.

    W~klatkach pośrednich analizowana jest odległość wszystkich punktów charakterystycznych od aktualnej \textit{trasy}. Dzięki temu, wiemy jak wygląda dystrybucja punktów charakterystycznych w~okół śledzonej trasy.

    Warto zauważyć, że taki sposób weryfikacji odporny jest na zmiany (poprzez dokładanie lub usuwanie) punktów charakerystycznych.

    Biorąc pod uwagę dwa wspomniane parametry tj. \textbf{odchylenie minimalne od punktu kluczowego w~danej klatce sekwencji wideo} oraz \textbf{wektor odchyleń punktów charakterystycznych od trasy} możemy skutecznie porównać jakość badanych algorytmów między sobą.

  \section{Procedura weryfikacji wydajności badanego algorytmu}\label{Section_Wydajnosc}

    W~celu analizy i~porównania wydajności poszczególnych algorytmów przygotowano zestaw parametrów, które obliczane będą podczas wykonania każdego algorytmu. Ujednolicenie to ma celu wypracowanie wspólnego wzorca porównawczego różnych algorytmów o~zupełnie innej zasadzie działania oraz implementacji.

    Główne parametry porównawcze opierają się na pomiarach czasu wykonania oraz własnościami statystycznymi wyciągniętymi z~pomiarów.

    Pierwszy pomar to \textbf{czas pracy algorytmu na pojedynczej klatce}. Z~wartości pomiarów dla wszystkich klatek sekwencji wideo wyciągnięte zostaną własności statystyczne tj. \textit{średnia arytmetyczna}, \textit{mediana}, \textit{odchylenie standardowe}.

    Kolejnym pomiarem jest \textbf{sumaryczny czas przetwarzania algorytmu}. Warto zauważyć, że nie będzie on nigdy mniejszy od nominalnego czasu trwania sekwencji wideo. Sumaryczny czas może być jedynie równy wartości nominalnej w~przypadku, gdy przetwarzanie pojedynczej klatki wideo odbędzie się w~czasie mniejszym bądź równym wartości niezbędnej do odczekania przed wyświetleniem następnej klatki. Czasu oczekiwania jest ściśle związany z~wartością \textit{częstotliwości klatek na sekundę} analizowanej sekwencji wideo.

    Następna wartość to \textbf{odchylenie sumarycznego czasu wykonania od czasu nominalnego} wyrażona w~procentach.

    Ostatni parametr to próba klasyfikacji wszystkich algorytmów według stopnia ich złożoności. Każdy algorytm będzie poddany analizie pod kątem wyselekcjonowanego zbioru operacji dominujących, w~celu wyznaczenia \textbf{złożoności obliczeniowej oraz pamięciowej}.