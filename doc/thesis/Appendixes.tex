\chapter{Środowisko uruchomieniowe}\label{Chapter_EnvironmentAppendix}

  \section{Zależności}\label{Section_Dependencies}

    Poniżej zebrane zostały zależności niezbędne do uruchomienia środowiska uruchomieniowego:
      \begin{itemize}
        \item Interpreter języka \textit{Python} w~wersji z~rodziny \textit{2.7.x},
        \item Narzędzie budowania \textit{SCons},
        \item Kompilator języka C++ (w~zależności od systemu operacyjnego \textit{GCC} lub \textit{Microsoft Visual Studio 2012 C++ Compiler}),
        \item Biblioteka \textit{OpenCV} w~wersji \textit{2.4.3} skompilowana pod odpowiedni system operacyjny,
        \item Środowisko uruchomieniowe \textit{node.js} w~wersji z~rodziny \textit{0.8.x},
        \item Współczesna i~aktualna przeglądarka internetowa (najlepiej \textit{Mozilla Firefox} lub \textit{Google Chrome}).
      \end{itemize}

  \section{Instalacja i~uruchomienie}\label{Section_Installation}

    Po zainstalowaniu niezbędnych środowisk, narzędzi oraz bibliotek dostarczonych wraz z~kodem źródłowym na nośniku dołączonym do pracy, należy skopiować katalog \textit{src} w~dowolne miejsce na dysku twardym, gdzie nasz użytkownik posiada prawa zapisu.

    Aby rozpocząć badanie, dla obu platform (\textit{Microsoft Windows} oraz \textit{Linux}), znajdując się w~powyższym katalogu należy wykonać z~linii poleceń komendę \textit{node run-tests.js}, który ustawia odpowiednie zmienne środowiskowe i~przystępuje do kompilacji oraz uruchomienia badań.

    Rezultaty badań po zakończeniu procesu znajdą się w~katalogu \textit{assets} wewnątrz uprzednio skopiowanego katalogu. Wykresy mogą zostać wygenerowane na żądanie w~interfejsie użytkownika, dostępnym za pomocą przeglądarki internetowej pod adresem \textit{http://localhost:9292/results.html}, po uprzednim uruchomieniu serwera aplikacji za pomocą komendy \textit{node server/server.js}.