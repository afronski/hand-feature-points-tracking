\chapter{Wstęp}\label{Chapter_Wstep}

\section{Motywacja}\label{Section_Motywacja}

\section{Cel}\label{Section_Cel}

\chapter{Problem śledzenia punktów charakterystycznych w sekwencjach wideo}

\section{Opis dziedziny problemu}\label{Section_Problematyka}
\cite{SalientPointsTracking05}
\cite{GoodFeaturesToTrack94}
\cite{LucasKanadeTracker81}

\section{Podstawowe definicje}\label{Section_Definicje}
\cite{SalientPointsTracking05}
\cite{GoodFeaturesToTrack94}
\cite{LucasKanadeTracker81}

\section{Analiza i porównanie wybranych algorytmów}\label{Section_Algorytmy}

\subsection{Algorytmy oparte punkty kluczowe i analizę wybitności}\label{Subsection_SaliencyAndKeypoints}
\cite{SalientPointsTracking05}

\subsection{Algorytm oparty o filtr Kalmana}\label{Subsection_Kalman}
\cite{KalmanFilter95}

\subsection{Algorytmy oparte o przepływ optyczny}\label{Subsection_OpticalFlow}
\textit{Lucas-Kanade}
\textit{Lucas-Kanade w wariacji Shi-Tomasi}
\textit{TV-L}
\textit{Optymalizacja}

\cite{LucasKanadeTracker81}
\cite{GoodFeaturesToTrack94}
\cite{TVL107}
\cite{OpticalFlowNonPriori05}

\subsection{Algorytmy oparte o filtr cząsteczkowy}\label{Subsection_ParticleFilter}
\cite{ParticleFilter05}
\cite{FingertipParticleFilter11}

\subsection{Algorytmy oparte o decyzyjne drzewa losowe}\label{Subsection_RandomizedTrees}
\cite{RandomizedTrees06}
\cite{TwoStageRandomizedTrees11}

\subsection{Algorytm oparty o analizę łańcuchów Markova}\label{Subsection_Vitterbi}
\cite{HandTrackingVitterbi05}

\chapter{Specyfikacja wewnętrzna}\label{Chapter_SpecyfikacjaWewnetrzna}

\section{Wykorzystane technologie}\label{Section_Technologie}

\subsection{C++}\label{Subsection_Cpp}
\textit{Standard ISO C++ z roku 2003}\cite{CppStroustrup}\cite{EffectiveCpp}\cite{MoreEffectiveCpp}

\textit{Krótkie wprowadzenie do Object Oriented Programming}

\textit{Zastosowanie OOP w wizji komputerowej}

\subsection{STL}\label{Subsection_STL}
\textit{Opis biblioteki STL}\cite{EffectiveStl}

\textit{Szablony}

\textit{Algorytmy}

\textit{Funktory}

\subsection{OpenCV}\label{Subsection_OpenCV}
\textit{Opis biblioteki OpenCV w wersji 2.0}\cite{LearningOpenCV}

\textit{Podstawowe koncepcje wykorzystywane przy pracy z OpenCV}\cite{OpenCVCookbook}

\subsection{Narzędzia pomocniczne}\label{Subsection_PomocniczeTechnologie}
\textit{R}

\textit{Automatyzacja pracy i koncepcja "ciągłej integracji"} - Bash, Scons, FFmpeg.

\textit{HTML5, CSS3 i "middle-end" stworzony za pomocą node.js}

\section{Szczegóły techniczne implementacji wybranych algorytmów}

\chapter{Specyfikacja przeprowadzonych badań}\label{Chapter_SpecyfikacjaPrzeprowadzonychBadan}

\section{Definicje gestów}\label{Section_DefinicjeGestow}

W~ramach przeprowadzonych badań stworzono definicje poszczególnych gestów wykonywanych dłońmi użytkownika. Specyfikacja ma na celu ujednolicenie oraz uwspólnienie pojęć potrzebnych przy przeprowadzeniu analizy jakościowej oraz wydajnościowej dla wykorzystywanych danych wejściowych.

Przed przystąpieniem do badań zostały stworzone testowe materiały zgodne z~protokołami zaprezentowanymi poniżej. W~zbiorze danych wejściowych możemy wyszczególnić następujące definicje gestów:

\begin{itemize}
  \item Okrąg wykonywany do kamery otwartą prawą dłonią (w pozycji wejściowej dłoń znajduje się na godzinie dwunastej, kierunek ruchu jest zgodny z ruchem wskazówek zegara, promień około 20cm).
  \item Krzyż równoramienny wykonywany zewnętrzną stroną dłoni z~wyciągniętymi dwoma palcami (wskazującym oraz środkowym) prawej dłoni (najpierw wykonywane jest ramię pionowe, następnie powracamy palcami do środka i kierujemy się w lewą stronę potem w prawą, promień krzyża to około 7cm).
  \item Oburęczny gest rozciągania dłoni od środka obrazu na odległość około 20cm (obie dłonie znajdują się na środku i są uformowane gest "szczypty", obie są złączone czubkami palców).
  \item Półokrąg znaczony wskazującym palcem lewej dłoni (poruszanie palcem i dłonią, kierunek przeciwny do ruchu wskazówek zegara, pozycja wyjściowa - palec wskazujący na godzinie dwunastej, promień 7cm).
  \item Litera L wykonywana całą prawą dłonią, kształt litery zarysowują czubki palców (poruszanie górną częścią dłoni, złączonymi palcami, wysokość litery około 7cm).
  \item Zaciśnięcie prawej dłoni do kamery od pozycji otwartej do zaciśniętej pięści bez poruszania dłoni. Pozycją wejściową jest rozłożona dłoń, w której każdy palec znajduje się osobno. W pozycji końcowej pozostaje zaciśnięta pięść z kciukiem na wierzchu. Gest powinien być wykonywany z różnymi prędkościami.
\end{itemize}

\section{Akwizycja i obróbka danych wejściowych}\label{Section_DefinicjeGestow}

\section{Procedura weryfikacji jakości badanego algorytmu}\label{Section_Jakosc}

\section{Procedura weryfikacji wydajności badanego algorytmu}\label{Section_Wydajnosc}

\chapter{Specyfikacja zewnętrzna}\label{Chapter_SpecyfikacjaZewnetrzna}

\chapter{Analiza rezultatów}\label{Chapter_AnalizaRezultatow}

\section{Wnioski}\label{Section_Wnioski}

\section{Proponowane usprawnienia}\label{Section_Usprawnienia}
\cite{HandOverFaceOcclusion07}

\chapter{Podsumowanie}\label{Section_Podsumowanie}