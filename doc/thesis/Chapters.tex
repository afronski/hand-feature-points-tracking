\chapter{Wstęp}\label{Chapter_Wstep}

\section{Motywacja}\label{Section_Motywacja}

\section{Cel}\label{Section_Cel}

\section{Opis wykorzystanych technologii}\label{Section_Technologie}

\subsection{C++}\label{Subsection_Cpp}
\textit{TODO: opis standardu ISO C++ 2003, wykorzystanie książek, opis OOP i jak zostało wykorzystane w tym przypadku}
\cite{CppStroustrup}
\cite{EffectiveCpp}
\cite{MoreEffectiveCpp}

\subsection{STL}\label{Subsection_STL}
\textit{TODO: opis bibloteki, podstawowych koncepcji - funktory, algorytmy}
\cite{EffectiveStl}

\subsection{OpenCV}\label{Subsection_OpenCV}
\textit{TODO: opis biblioteki w wersji 2.0 + podstawowe koncepcje wykorzystywane przy pracy z biblioteką}
\cite{LearningOpenCV}
\cite{OpenCVCookbook}

\subsection{Pomocnicze technologie}\label{Subsection_PomocniczeTechnologie}
\textit{TODO: R (analiza statystyczna), FFmpeg, SCons, bash (koncepcja prostego CI i pełnej automatyzacji), node.js, CSS3, HTML5 (pomocnicze narzędzia)}

\chapter{Specyfikacja wewnętrzna}\label{Chapter_SpecyfikacjaWewnetrzna}

\section{Opis problematyki}\label{Section_Problematyka}
\cite{TrackingClusteringGpu10}

\section{Podstawowe definicje}\label{Section_Definicje}
\cite{SalientPointsTracking05}

\section{Analiza wybranych algorytmów}\label{Section_Algorytmy}

\subsection{}\label{Subsection_}
\cite{SalientPointsTracking05}

\subsection{}\label{Subsection_}
\cite{OpticalFlowNonPriori05}

\subsection{}\label{Subsection_}
\cite{HandTrackingVitterbi05}

\subsection{}\label{Subsection_}
\cite{FingertipParticleFilter11}
\cite{ParticleFilter05}

\subsection{}\label{Subsection_}
\cite{TwoStageRandomizedTrees11}
\cite{RandomizedTrees06}

\chapter{Specyfikacja przeprowadzonych badań}\label{Chapter_SpecyfikacjaPrzeprowadzonychBadan}

\section{Definicje gestów}\label{Section_DefinicjeGestow}

W~ramach przeprowadzonych badań stworzono definicje poszczególnych gestów wykonywanych dłońmi użytkownika. Specyfikacja ma na celu ujednolicenie oraz uwspólnienie pojęć potrzebnych przy przeprowadzeniu analizy jakościowej oraz wydajnościowej dla wykorzystywanych danych wejściowych.

Przed przystąpieniem do badań zostały stworzone testowe materiały zgodne z~protokołami zaprezentowanymi poniżej. W~zbiorze danych wejściowych możemy wyszczególnić następujące definicje gestów:

\begin{itemize}
  \item Okrąg wykonywany do kamery otwartą prawą dłonią (w pozycji wejściowej dłoń znajduje się na godzinie dwunastej, kierunek ruchu jest zgodny z ruchem wskazówek zegara, promień około 20cm).
  \item Krzyż równoramienny wykonywany zewnętrzną stroną dłoni z~wyciągniętymi dwoma palcami (wskazującym oraz środkowym) prawej dłoni (najpierw wykonywane jest ramię pionowe, następnie powracamy palcami do środka i kierujemy się w lewą stronę potem w prawą, promień krzyża to około 7cm).
  \item Oburęczny gest rozciągania dłoni od środka obrazu na odległość około 20cm (obie dłonie znajdują się na środku i są uformowane gest "szczypty", obie są złączone czubkami palców).
  \item Półokrąg znaczony wskazującym palcem lewej dłoni (poruszanie palcem i dłonią, kierunek przeciwny do ruchu wskazówek zegara, pozycja wyjściowa - palec wskazujący na godzinie dwunastej, promień 7cm).
  \item Litera L wykonywana całą prawą dłonią, kształt litery zarysowują czubki palców (poruszanie górną częścią dłoni, złączonymi palcami, wysokość litery około 7cm).
  \item Zaciśnięcie prawej dłoni do kamery od pozycji otwartej do zaciśniętej pięści bez poruszania dłoni. Pozycją wejściową jest rozłożona dłoń, w której każdy palec znajduje się osobno. W pozycji końcowej pozostaje zaciśnięta pięść z kciukiem na wierzchu. Gest powinien być wykonywany z różnymi prędkościami.
\end{itemize}

\section{Procedura weryfikacji jakościowej}\label{Section_ProceduraWeryfikacjiJakosciowej}

\section{Procedura weryfikacji wydajnościowej}\label{Section_ProceduraWeryfikacjiWydajnosciowej}

\chapter{Specyfikacja zewnętrzna}\label{Chapter_SpecyfikacjaZewnetrzna}

\chapter{Analiza rezultatów}\label{Chapter_AnalizaRezultatow}

\section{Proponowane usprawnienia}\label{Section_Usprawnienia}
\cite{HandOverFaceOcclusion07}

\section{Wnioski}\label{Section_Wnioski}