\chapter{Wstęp}\label{Chapter_Wstep}

\section{Cel}\label{Section_Cel}

\chapter{Specyfikacja przeprowadzonych badań}\label{Chapter_SpecyfikacjaPrzeprowadzonychBadan}

\section{Definicje gestów}\label{Section_DefinicjeGestow}

W~ramach przeprowadzonych badań stworzono definicje poszczególnych gestów wykonywanych dłońmi użytkownika. Specyfikacja ma na celu ujednolicenie oraz uwspólnienie pojęć potrzebnych przy przeprowadzeniu analizy jakościowej oraz wydajnościowej dla wykorzystywanych danych wejściowych.

Przed przystąpieniem do badań zostały stworzone testowe materiały zgodne z~protokołami zaprezentowanymi poniżej. W~zbiorze danych wejściowych możemy wyszczególnić następujące definicje     getstów:

\begin{itemize}
  \item Okrąg wykonywany do kamery otwartą prawą dłonią.
  \item Krzyż równoramienny wykonywany zewnętrzną stroną dłoni z~wyciągniętymi dwoma palcami (wskazującym oraz środkowym) prawej dłoni.
  \item Oburęczny gest rozciągania od środka obrazu na odległość około 20cm (obie dłonie znajdują się na środku i są uformowane gest "szczypty").
  \item Półokrąg znaczony wskazującym palcem lewej dłoni.
  \item Litera L wykonywana całą prawą dłonią, kształt litery zarysowują czubki palców.
\end{itemize}

\section{Procedura weryfikacji jakościowej}\label{Section_ProceduraWeryfikacjiJakosciowej}

\section{Procedura weryfikacji wydajnościowej}\label{Section_ProceduraWeryfikacjiWydajnosciowej}