\chapter{Wstęp}\label{Chapter_Wstep}

  \section{Motywacja}\label{Section_Motywacja}

  \section{Cel}\label{Section_Cel}

\chapter{Problem śledzenia punktów charakterystycznych w~sekwencjach wideo}

  \section{Opis dziedziny problemu}\label{Section_Problematyka}
    \cite{SalientPointsTracking05}
    \cite{GoodFeaturesToTrack94}
    \cite{LucasKanadeTracker81}

  \section{Podstawowe definicje}\label{Section_Definicje}
    \cite{SalientPointsTracking05}
    \cite{GoodFeaturesToTrack94}
    \cite{LucasKanadeTracker81}

  \section{Analiza i~porównanie wybranych algorytmów}\label{Section_Algorytmy}

    \subsection{Algorytmy oparte punkty kluczowe i~analizę wybitności}\label{Subsection_SaliencyKeypoints}
      \cite{SalientPointsTracking05}

    \subsection{Algorytmy oparte o~przepływ optyczny}\label{Subsection_OpticalFlow}

      \subsubsection{Ogólny zarys algorytmów opartych o~przepływ optyczny}
      \cite{OpticalFlowNonPriori05}
      \cite{LucasKanadeTracker81}

      \subsubsection{Lucas-Kanade}
      \cite{OpenCvOpticalFlow04}

      \subsubsection{Horn-Shunck}
      \cite{HornSchunck81}

      \subsubsection{TV-L}
      \cite{TVL107}

      \subsubsection{Problem doboru cech do śledzenia dla algorytmów opartych o przepływ optyczny}
      \cite{GoodFeaturesToTrack94}

    \subsection{Algorytm oparty o~filtr cząsteczkowy}\label{Subsection_ParticleFilter}
      \cite{ParticleFilter05}
      \cite{FingertipParticleFilter11}

    \subsection{Algorytm oparty o~filtr Kalmana}\label{Subsection_Kalman}
      \cite{KalmanFilter95}

    \subsection{Algorytm oparty o~decyzyjne drzewa losowe}\label{Subsection_RandomizedTrees}
      \cite{RandomizedTrees06}
      \cite{TwoStageRandomizedTrees11}

    \subsection{Algorytm oparty o~analizę łańcuchów Markova}\label{Subsection_Vitterbi}
      \cite{HandTrackingVitterbi05}

\chapter{Specyfikacja przeprowadzonych badań}\label{Chapter_SpecyfikacjaPrzeprowadzonychBadan}

  \section{Definicje gestów}\label{Section_DefinicjeGestow}

    W~ramach przeprowadzonych badań stworzono definicje poszczególnych gestów wykonywanych dłońmi użytkownika. Poniższa specyfikacja ma na celu uwspólnienie pojęć potrzebnych przy przeprowadzeniu analizy jakościowej oraz wydajnościowej.

    Wytyczne były pomocne przy procesie akwizycji oraz obróbki danych wejściowych. Wszystkie stworzone materiały testowe są zgodne z~przedstawionymi protokołami. W~zbiorze danych wejściowych możemy wyróżnić następujące definicje gestów:

    \gest{Okrąg}
         {Gest wykonywany na tle twarzy lub ciała.}
         {Okrąg o promieniu 20 cm.}
         {Prawa dłoń.}
         {Okrąg wykonywany do kamery otwartą dłonią.}
         {Kierunek ruchu jest zgodny z~ruchem wskazówek zegara.}
         {Dłoń znajduje się na godzinie dwunastej.}
         {Punkty charakterystyczne rozmieszczone równomiernie na krawędziach dłoni.}
         {Punkty kluczowe połączone linią ciągłą rozmieszczone na okręgu co 30\degree.}

    \newpage
    \gest{Krzyż}
         {Gest wykonywany na tle twarzy lub ciała.}
         {Krzyż równoramienny o~promieniu 7 cm.}
         {Dwa, wyciągnięte palce (wskazujący oraz środkowy) prawej dłoni}
         {Kształt zarysowywany zewnętrzną strona dłoni.}
         {W~pierwszej kolejności wykonywane jest ramię pionowe, następnie palce powracają do środka i~gest wykonywany jest w~lewą stronę a~następnie w~prawo.}
         {Ustawiona dłoń znajduje się na godzinie dwunastej.}
         {Punkty charakterystyczne rozmieszczone na czubkach wyciągniętych palców.}
         {5 punktów kluczowych - cztery na krawędziach i~jeden w~środku figury.}

    \gest{Rozszerzanie}
         {Gest wykonywany w~pozycji siedzącej.}
         {W~pozycji wyjściowej, jest to kwadrat o~długości boku około 20 cm.}
         {Obie dłonie złączone czubkami palców uformowane w~gest \textit{szczypty}.}
         {Oburęczny gest rozciągania dłoni od środka.}
         {Gest powinien symulować rozciąganie niewidzialnego materiału. Kształt dłoni nie powinien ulegać zmianie w~trakcie wykonywania gestu.}
         {Obie złączone dłonie znajdują się w~środku figury na wysokości klatki piersiowej.}
         {Punkty charakterystyczne rozmieszczone równomiernie na powierzchni obu dłoni.}
         {Punkty kluczowe równomiernie rozmieszczone na przekątnej kwadratu.}

    \newpage
    \gest{Duża litera C}
         {Gest wykonywany na jednolitej i~gładkiej powierzchni.}
         {Półokrąg o~promieniu 10 cm.}
         {Palec wskazujący lewej dłoni.}
         {Zarysowanie kształtu w~kierunku przeciwnym do ruchu wskazówek zegara.}
         {Gest powinien być wykonywany przez poruszenie całej dłoni, nie tylko samego palca.}
         {Palec wskazujący na godzinie dwunastej.}
         {Punkty charakterystyczne zaznaczone na krawędziach analizowanego palca.}
         {Punkty połączone linią ciągłą rozmieszczone równomiernie na półokręgu co 30\degree.}

    \gest{Duża litera L}
         {Gest wykonywany na jednolitej i~gładkiej powierzchni.}
         {Litera "L" o~wysokości 7 cm.}
         {Złączone palce prawej dłoni.}
         {Kształt litery zarysowują czubki palców.}
         {Gest powinien być wykonywany przez poruszanie tylko złączonymi palcami, nie samą dłonią.}
         {Palce na godzinie dwunastej.}
         {Punkty charakterystyczne rozmieszczone równomiernie na górnej części dłoni.}
         {4 punkty kluczowe - na początku, na końcu, na zgięciu litery oraz dodatkowy punkt na dłuższej części litery, dokładnie w~jej połowie.}

    \newpage
    \gest{Zgniecenie}
         {Gest wykonywany na tle twarzy lub ciała.}
         {Zaciśnięcie dłoni od rozwarcia do pięści.}
         {Prawa dłoń.}
         {W~pozycji wejściowej otwarta dłoń, w~pozycji końcowej pięść zwrócona palcami do kamery, z~kciukiem na wierzchu.}
         {Gest powinien być wykonywany bez poruszania dłonią, z~jednolitą prędkością.}
         {Otwarta dłoń z~rozłożonymi palcami, skierowana wewnętrzną stroną do kamery.}
         {Punkty charakterystyczne znajdują się na czubkach palców.}
         {Punkty kluczowe rozmieszczone równomiernie na drodze wykonywanej przez każdy z~palców z~osobna.}

  \section{Akwizycja i~obróbka danych wejściowych}\label{Section_DefinicjeGestow}

  \section{Procedura weryfikacji jakości badanego algorytmu}\label{Section_Jakosc}

  \section{Procedura weryfikacji wydajności badanego algorytmu}\label{Section_Wydajnosc}


\chapter{Specyfikacja wewnętrzna}\label{Chapter_SpecyfikacjaWewnetrzna}

  \section{Wykorzystane technologie}\label{Section_Technologie}

    \subsection{C++}\label{Subsection_Cpp}

      \subsubsection{Standard ISO C++ z~roku~2003}
        \cite{CppStroustrup}\cite{EffectiveCpp}\cite{MoreEffectiveCpp}

      \subsubsection{Object Oriented Programming}

      \subsubsection{Zastosowanie \textit{OOP} w~wizji komputerowej}

    \subsection{STL}\label{Subsection_STL}

      \subsubsection{Opis biblioteki}
        \cite{EffectiveStl}

      \subsubsection{Szablony}

      \subsubsection{Algorytmy}

      \subsubsection{Funktory}

    \subsection{OpenCV}\label{Subsection_OpenCV}

      \subsubsection{Opis biblioteki}
        \cite{LearningOpenCV}

      \subsubsection{Podstawowe koncepcje wykorzystywane przy pracy z OpenCV}
        \cite{OpenCVCookbook}

    \subsection{Narzędzia pomocniczne}\label{Subsection_PomocniczeTechnologie}

      \subsubsection{R}

      \subsubsection{Automatyzacja pracy i~koncepcja "ciągłej integracji"}
        Bash, Scons, FFmpeg.

      \subsubsection{HTML5, CSS3 i~warstwa pośrednia oparta o~node.js}

  \section{Szczegóły techniczne implementacji wybranych algorytmów}

\chapter{Specyfikacja zewnętrzna}\label{Chapter_SpecyfikacjaZewnetrzna}

\chapter{Analiza wyników badań}\label{Chapter_AnalizaRezultatow}

  \section{Wyniki weryfikacji jakościowej badanych algorytmów}\label{Section_Quality}

  \section{Wnioski badań jakości}\label{Section_QualityWnioski}

  \section{Wyniki weryfikacji wydajnościowej badanych algorytmów}\label{Section_Performance}

  \section{Wnioski badań wydajności}\label{Section_PerformanceWnioski}

  \section{Proponowane usprawnienia}\label{Section_Usprawnienia}
    \cite{HandOverFaceOcclusion07}

\chapter{Podsumowanie}\label{Section_Podsumowanie}